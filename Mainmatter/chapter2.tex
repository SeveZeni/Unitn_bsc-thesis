\chapter{GHZ theorem}
\label{chap:ghz-theorem}
In this chapter we will present an argument to show that the criticism leveled against quantum mechanics by EPR is void.

The first result of this type is due to John S. Bell and is presented in his work of 1964 \cite{Bell1964}. This thesis will not be concerned with Bell's reasoning but will rather illustrate a more recent argument that leads to the conclusion that EPR's premises are inconsistent when applied to systems of three particles. This argument was proposed by Daniel M. Greenberg, Michael A. Horne and Anton Zeilinger in 1989 \cite{ghz1989} (see also \cite{:/content/aapt/journal/ajp/58/12/10.1119/1.16243}).


\section{Summary and formalization of EPR results}
This section serves us to introduce the notation we will use to expose the main matter of this chapter, no new result will be derived here.

In the previous chapter starting from the assumptions of perfect correlation (\ref{itm:epr-perfect-correlation}), completeness (\ref{itm:epr-completeness}), reality (\ref{itm:epr-reality}) and locality (\ref{itm:epr-locality}) we proved the following results:
\begin{enumerate}
  %enumerated lists style
  \renewcommand{\theenumi}{\roman{enumi}}
  \renewcommand{\labelenumi}{(\theenumi)}
\item \label{itm:incompleteness} \textit{Incompleteness:} there are elements of the physical reality that have no counterpart in quantum mechanics.
\item \label{itm:determinism} \textit{Determinism:} the result of a measurement of a spin component (of one of two particles in the state (\ref{eq:singlet-state})) depends on those elements of the physical reality and is predetermined (i.e. no indeterministic process takes place when the measurement is performed).%should i specify spin component of _entangled_ particle?
\end{enumerate}
Result (\ref{itm:incompleteness}) implies that there exists a more complete specification of the physical reality. Let us denote by $\lambda$ this more complete description of a pair of particles compatible with the perfect correlation assumption (\ref{itm:epr-perfect-correlation}), i.e. a description that comprises the elements of physical reality we have found to exist and are ignored by quantum mechanics. Quoting Bell's article \cite{Bell1964}, in which he introduced for the first time this notation: ``It is a matter of indifference ... whether $\lambda$ denotes a single variable or a set, or even a set of functions, and whether the variables are discrete or continuous.'' By (\ref{itm:determinism}) we then deduce (again following Bell \cite{Bell1964}) that the result of a spin component measurement on particle 1, $A$, is a function of $\lambda$, that is:
\begin{equation}
  A = A_i(\lambda)
  \label{eq:results-particle-1}
\end{equation}
where $i = x, y$ denotes the spin component (of particle 1) measured. Analogously, for particle 2, the outcome of a measurement of the $j = x, y$ spin component will be given by the function:
\begin{equation}
 B  = B_j(\lambda).
 \label{eq:results-particle-2}
\end{equation}
%is it clear enough the dependence on the settings of the apparatus?

In writing $A_i(\lambda)$ and $B_j(\lambda)$ this way we have implicitly fulfilled a request that comes from locality, that is: the result of a measurement of a spin component on particle 1 does not depend on the spin component of particle 2 that is measured (i.e. $A$ depends on $i$ only and not on $j$), conversely $B$ only depends on $j$ and not on $i$.%is some better explenation needed here?

\begin{remark}
  \label{rem:well-def-ab}
  It is easy to see that the perfect correlation assumption ``only'' implies that (\ref{eq:results-particle-1}) and (\ref{eq:results-particle-2}) hold on a subset of $\lambda$'s of measure unity. We will come back to this in due time.
\end{remark}

\begin{observation}
Working on the objects just presented Bell \cite{Bell1964} could derive a contradiction between the EPR assumptions ((\ref{itm:epr-perfect-correlation}), (\ref{itm:epr-completeness}), (\ref{itm:epr-reality}) and (\ref{itm:epr-locality})) and statistical predictions of quantum mechanics. We will not present his argument here as we will focus on a more recent result on the same matter.
\end{observation}


\section{Adaptation of the previous results to systems of three spin-1/2 particles}
\label{sec:adaptation-to-3-particles}
Before moving on to present the main matter of this chapter we need to adapt some of the results obtained so far to systems consisting of more that two particles. In particular we will be concerned with systems of three entangled particles and thus we will limit our attention to this case.%is entangled relevant here?

Consider a system of three spin-1/2 particles in the spin state:
\begin{equation}
  |\chi\rangle = \frac{1}{\sqrt{2}} \left( |+\rangle_1 |+\rangle_2 |+\rangle_3 - |-\rangle_1 |-\rangle_2 |-\rangle_3 \right)
  \label{eq:ghz-state}
\end{equation}
where $|+\rangle_i$ and $|-\rangle_i$ are states of spin up and down along the $\mathbf{\hat{z}}$ direction respectively of particle $i = 1, 2, 3$.
We are interested in the predictions of quantum mechanics for the results of experiments that measure a spin component of each particle, in particular we are interested in the following observables:
\begin{equation}
  \begin{split}
    S_{1x} \times S_{2y} \times S_{3y}\\
    S_{1y} \times S_{2x} \times S_{3y}\\
    S_{1y} \times S_{2y} \times S_{3x}
  \end{split}
  \label{eq:xyy-observables}
\end{equation}
It is easy to show that for all of these quantum mechanics predicts with certainty (i.e. with probability equal to 1) the result $+ 1$ (the proof of this statement is given in Appendix \ref{app:spin-predictions}).

Once again we can observe that the result of a measurement on one of the particles can be predicted with certainty by an appropriate measurement on the other two particles (perfect correlation). If the particles are emitted at a source and move away from it in different directions, let's say in the $xy$ plane at $120^{\circ}$ form each other, the measurements can be made in places very far from each other. We are thus in a situation analogous to that envisaged by EPR for their argument and the assumptions (\ref{itm:epr-perfect-correlation}), (\ref{itm:epr-completeness}), (\ref{itm:epr-reality}) and (\ref{itm:epr-locality}) can now be written in the form:%is it really just a form issue?

\begin{enumerate}[label=(\alph*$'$)]
  %enumerated lists style
\item \label{itm:epr-perfect-correlation'} \textit{Perfect correlation:} If we measure the $S_y$ spin component of any two of the three particles and the $S_x$ component of the remaining one then, the result of the measurement on one of the particles can be predicted with certainty by performing the measurements on the other two.
\item \label{itm:epr-completeness'} \textit{Completeness:} Same as (\ref{itm:epr-completeness}).
\item \label{itm:epr-reality'} \textit{Reality:} Same as (\ref{itm:epr-reality}).
\item \label{itm:epr-locality'} \textit{Locality:} Since at the time of measurement the three particles no longer interact, no real change can take place in one particle in consequence of anything that may be done to the other two.
\end{enumerate}
The EPR argument presented in section \ref{epr-argument} can then be adapted to these assumptions and thus we conclude that in this case too there exists a more complete specification of the physical reality. If once again we denote this more complete specification by $\lambda$ we also conclude that the results of spin component measurements on particles 1, 2 and 3, respectively $A$, $B$ and $C$, are functions of the spin component being measured and of $\lambda$:%dependence on component to uniform with previous section
\begin{equation}
  \label{eq:results-abc}
  \begin{split}
      A = A_i(\lambda)\\
      B = B_j(\lambda)\\
      C = C_k(\lambda)
  \end{split}
\end{equation}
where $i, j, k = x, y$ denote the spin component.%is better wording needed?

In this case also, locality demands that $A$ doesn't depend on the spin components of particles 2 and 3 that are measured, i.e. $A$ cannot depend on $j$ and $k$. Similarly $B$ cannot depend on $i$ and $k$ and $C$ cannot depend on $i$ and $j$.

\begin{remark}
  \label{rem:well-def-abc}
  The same as in Remark \ref{rem:well-def-ab} holds for the functions in (\ref{eq:results-abc}).
\end{remark}


\section{GHZ argument}
We are now ready to prove the main result of this chapter. With the machinery introduced so far it will be a matter of a few lines.

We have seen in the previous section that the state (\ref{eq:ghz-state}) is an eigenstate of the three observables (\ref{eq:xyy-observables}) with eigenvalue $+ 1$. The EPR argument adapted to the case of three particles in state (\ref{eq:ghz-state}) led us to the conclusion that there exist a more complete specification of the physical reality, $\lambda$, and three functions $A$, $B$ and $C$ that determine the results of spin component measurements on the particles. Thus (assuming again that quantum mechanics gives correct predictions):%is the part in brackets needed? GHZ comment on Bell: since EPR's argument for their program commenced with the perfect quantum mechanical correlations, it is essential that the expectation value of q.m. agrees with that of local-realism
\begin{equation}
  \begin{split}
    A_x B_y C_y = 1\\
    A_y B_x C_y = 1\\
    A_y B_y C_x = 1
  \end{split}
  \label{eq:xyy-local-realist-results}
\end{equation}
where we have dropped the $\lambda$ in the notation because we are holding it fixed from now on.

If we considered a measurement of the $S_x$ spin component on all three particles the result of such a measurement would be given by $A_x B_x C_x$. Because $A_y^2 = 1$, and the same goes for $B$ and $C$, this can be obtained by multiplying together the three equations in (\ref{eq:xyy-local-realist-results}) yielding:
\begin{equation}
  A_x B_x C_x = 1.
  \label{eq:xxx-local-realist-result}
\end{equation}
However, the quantum mechanical prediction for the observable:
\begin{equation}
  S_{x1} \times S_{2x} \times S_{3x},
  \label{eq:xxx-observable}
\end{equation}
on the state of equation (\ref{eq:ghz-state}), is $- 1$ with probability equal to 1 (refer to Appendix \ref{app:spin-predictions}) in contrast to equation (\ref{eq:xxx-local-realist-result}).

We have thus obtained a contradiction that can be summarized in the following:
\begin{theorem}
  One (at least) of the following statements is false:
  \begin{enumerate}[label=(\roman*)]
  \item \label{itm:ghz-theorem-locality} Nature is compatible with assumption \ref{itm:epr-locality'}
  \item \label{itm:ghz-theorem-reality} Nature is compatible with the notion of ``element of physical reality'' implied by \ref{itm:epr-reality'}.
  \item Quantum mechanics gives correct predictions for perfect correlations.
  \end{enumerate}
\end{theorem}

\begin{remark}
  We have seen in Remark \ref{rem:well-def-abc} that the functions $A$, $B$ and $C$ need to be well defined ``only'' on a subset of $\lambda$'s of measure unity. However, since finite unions of sets of measure zero also have measure zero, the equations in (\ref{eq:xyy-local-realist-results}) and (\ref{eq:xxx-local-realist-result}) also hold on a set of $\lambda$'s of measure one.
\end{remark}

%\begin{observation}
%  We could try to investigate further where exactly the conflict between quantum mechanics and the requests of locality and reality arises.
%  HERE GOES A COMMENT ABOUT WHAT THE ROLE OF LOCALITY AND REALITY HAVE IN THE ARGUMENT. LALOE PG 40.
%\end{observation}

\begin{note}%don't like where this note is.
  \ref{itm:ghz-theorem-locality} and \ref{itm:ghz-theorem-reality} together are referred to as local-realism.
\end{note}
