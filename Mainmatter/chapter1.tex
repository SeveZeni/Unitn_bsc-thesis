\chapter{The EPR program}

The first section of this chapter will concern the argument brought foreward by Einstein, Podolsky and Rosen in their article \cite{PhysRev.47.777}, in particular the adaptation of the argument proposed by Bohm in \cite{bohm1951quantum} will be treated (see also \cite{PhysRev.108.1070}).
It will be shown that under reasonable assumptions, regarding the locality of physical processes and reality, the quantum description of reality is incomplete (i.e. there exists \textit{elements of physical reality} for which there is no couterpart in the formalism). TO REVISE

In the last section of the chapter some of Bell's contributions on the matter will be presented. In particular...


\section{EPR ``theorem''}

\begin{figure}
  \centering
  \includegraphics[width=0.25\textwidth]{Mainmatter/Chapter1/eprb.png}
  \caption{Bohm's adaptation of EPR's gedankenexperiment. Two spin-1/2 particles  emitted by the source (in the singlet state of spin)...}
  \label{fig:eprb-gedankenexp}
\end{figure}

\subsection*{EPRB gedankenexperiment}

Consider a system composed of two spin-1/2 particles flying off a source in opposite directions as shown in Fig. \ref{fig:eprb-gedankenexp}. Suppose that the two particles are emitted by the source in the state with total spin equal to $0$, that is in the state:
\begin{equation}
  |\Psi\rangle = \frac{1}{\sqrt{2}} \left( |S_z = + 1\rangle_1 |S_z = - 1\rangle_2 - |S_z = - 1\rangle_1 |S_z = + 1\rangle_2 \right)
  \label{eq:singlet-state}
\end{equation}
Once the two particles are far from each other they enter two apparatuses which can measure either $S_x$ or $S_y$ (e.g. two Stern-Gerlach apparatuses with magnetic fields oriented either in the $\hat{\textbf{x}}$ or $\hat{\textbf{y}}$ direction).

Let us see what the prediction of quantum mechanics for such an experiment are, in particular we are interested in the (perfect) correlation between the result of the measurement performed on the particle propagating in the positive $\hat{\textbf{z}}$ direction (which we will call particle A from now on) with that on the particle propagating in the negative $\hat{\textbf{z}}$ direction (particle B).

REPHRASING NEEDED

Suppose that both apparatuses are set to measure $S_x$, to obtain the prediction it is convenient to express the state (\ref{eq:singlet-state}) on a basis of eigenvectors of the $S_x$ observable. It turns out that the state in question takes the same form on any basis of the spin space (see Appendix \ref{app:spin-rotations}), and thus:
\begin{equation*}
  |\Psi\rangle = \frac{1}{\sqrt{2}} \left( |S_x = + 1\rangle_1 |S_x = - 1\rangle_2 - |S_x = - 1\rangle_1 |S_x = + 1\rangle_2 \right)
\end{equation*}
We can see that if the particle propagating in the positive $\hat{\textbf{z}}$ direction is found to have $S_x = + 1$ ($S_x = - 1$) then the other particle will be found to have $S_x = - 1$ ($S_x = + 1$).
We have just said that the state (\ref{eq:singlet-state}) takes the same form on any basis of the spin space, thus if both apparatuses are set to measure $S_y$ we obtain analogous results.

\subsection*{EPR argument}

In the previous paragaph we have only presented some aspects of the quantum mechanical description of systems composed of two spin-1/2 particles. We are now ready to present the EPR argument.
Let us start by listing the assumptions on which the argument rests:
\begin{enumerate}
  %enumerated lists style
  \renewcommand{\theenumi}{\alph{enumi}}
  \renewcommand{\labelenumi}{(\theenumi)}
\item \label{itm:perfect-correlations} \textit{Perfect correlation:} If the spins of the two particles are measured along the same direction, then the two spins will be found (with certainity) to have opposite sign.
\item \label{itm:epr-locality} \textit{Locality:} ``Since at the time of measurement the two systems no longer interact, no real change can take place in the second system in consequence of anything that may be done to the first system.''
\item \label{itm:epr-reality} \textit{Reality:} ``If, without in any way disturbing a system, we can predict with certainity (i.e. with probability equal to unity) the value of a physical quantity, then there exist an element of physical reality corresponding to this physical quantity.'' ADD SOMETHING HERE
\item \label{itm:epr-completeness} \textit{Completeness:} ``Every element of the physical reality must have a counterpart in the [complete] physical theory.''
\end{enumerate}
The first of these assumptions, as we have seen, is a prediction of quantum mechanics. The three latter ones are reasonable propositions about locality, reality and completeness of a physical theory which we have quoted from the original article by Einstein, Podolsky and Rosen \cite{PhysRev.47.777}.

The agrument is then the following:
\begin{enumerate*}
\item Perfect correlations, (\ref{itm:perfect-correlations}), allow us to predict \textit{with certainity} the outcome of a spin component measurement on the particle B by first measuring the same spin component on particle A;
\item Because of locality, (\ref{itm:epr-locality}), the measurement performed on particle A cannot cause a change on particle B.
\item By reality, (\ref{itm:epr-reality}), we can state that the spin component considered is an element of physical reality.
\item We observe that the previous points in the argument hold for both $S_x$ and $S_y$, we must thus admit that the two spin components are both \textit{elements of physical reality} with a definite value. Since there is no state in quantum mechanics in which both spins have definite value we conclude that the quantum mechanicsl description of reality is incomplete.
\end{enumerate*}

THE WORD ENTANGLEMENT NEVER COMES UP

\begin{observation}
The agrument presented here is not the original argument that Einstein, Podolsky and Rosen present in their article \cite{PhysRev.47.777}. The difference being...
\end{observation}

\begin{observation}
  From the results derived in Appendix \ref{app:spin-rotations} it can easily be seen that the argument is more general. In particular it can be shown that the result is not limited to the $\hat{\bftext(x}}$ and $\hat{\bftext(y}}$ components of spin and the argument applies to any component of spin. Hence any component of spin is an \textit{element of physical reality}
\end{observation}

\subsection{Consequences of EPR's result}
The EPR argument 

They don't give a completation

\section{How Bell introduced EPR elements of reality into the formalism}


Bell could derive experimentally testable inequalities... ...we are not going to present here (references)
