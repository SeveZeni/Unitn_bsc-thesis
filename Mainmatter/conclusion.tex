\chapter*{Conclusion}
\addcontentsline{toc}{chapter}{Conclusion}
In this thesis we have presented EPR's critique of quantum mechanics, we have seen that, under the assumptions of perfect correlation, reality and locality, quantum mechanics is not a complete description of reality (according to EPR's criterion of completeness).

We briefly mentioned the Bell's inequality, that can be derived from the EPR assumptions, and we have seen that the inequality is violated by statistical predictions of quantum mechanics.

Then we have illustrated the GHZ argument showing that the EPR reasoning doesn't apply to systems of three particles. We have seen that for such systems the EPR premises are inconsistent and, in contrast with Bell's argument, the contradiction arises for perfect correlations predicted by quantum mechanics\footnote{In \cite{:/content/aapt/journal/ajp/58/12/10.1119/1.16243} the authors note: ``There is an irony in this result in that perfect correlations are central to EPR's argument for the existence of states more complete than those of quantum mechanics.''}. Hence, the demonstration of the conflict between EPR assumptions and quantum mechanics doesn't involve inequalities.

In the last chapter we have seen that, in principle, the GHZ reasoning allows to design experiments in which a single run (of the experiment) is sufficient to perform a test of quantum mechanics against local-realism. To our knowledge no experiment of this type has been performed.

Finally, we have discussed some actual experimental effort towards a test of quantum mechanics following the GHZ argument. We have presented an experiment that has been performed to observe GHZ entanglement, but the analysis we have reported didn't allow us to make claims in favor or against quantum mechanics. However, we have given reference to a more refined analysis of a slightly modified version of the experiment that allowed such claims to be made.

The techniques employed in the experiment we have presented also allowed us to observe that entanglement doesn't necessarily arise from interaction of the entangled subsystems but can also be obtained by a suitable projection (onto an entangled state) of a factorized state.
