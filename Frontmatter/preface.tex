\chapter*{Preface}
Quantum mechanics describes physical systems by means of a state vector, $| \psi \rangle$. For systems extended in space a description of a portion of the system (located in some space region) cannot in general be given without taking into account all other constituents of the system (no matter how far these are). This is at the basis of the phenomenon of entanglement.

In 1935 Albert Einstein, Boris Podolsky and Natan Rosen (EPR) realized that entanglement allowed a step further in the discussion on the foundations of quantum mechanics. In particular it allowed them to show that, under reasonable assumptions, $| \psi \rangle$ could not be regarded as a satisfactory description of single physical systems. Under these assumptions, $| \psi \rangle$ could provide at most correct description of statistical ensembles of systems.

The next milestone in the discussion was the contribution by John S. Bell. In 1964 he showed that the premises of the EPR argument lead to an inequality (Bell's inequality) that is violated by statistical predictions of quantum mechanics. This inequality allows experimental tests of the quantum mechanical predictions against those descending from EPR's assumptions. Since the seventies a variety of such tests have been made with no one being able to disprove quantum mechanics.

This thesis will be concerned with some elements of this debate. In chapter \ref{chap:epr} we will present the EPR argument. Bell's reasoning will not be illustrated in favor of a more recent result on the same matter that Daniel M. Greenberg, Michael A. Horne and Anton Zeilinger (GHZ) proposed in 1989 (chapter \ref{chap:ghz-theorem}). Finally, chapter \ref{chap:ghz-experiments} will be concerned with some of the experimental efforts that have been made towards a test of quantum mechanics following the argument by GHZ.

We will see that the GHZ argument will lead us to conclusions analogous to those obtained by Bell but in a way that doesn't involve inequalities. We will also see, that from an experimental point of view, the argument allows, in principle, to design experiments in which a single run suffices to confirm or disprove quantum mechanics. This is in contrast with experiments that follow Bell's argument where statistical considerations on many runs of the experiment must be done.
